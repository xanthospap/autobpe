\subsection{ddprocess}
\label{ddprocess}

\subsubsection{Purpose}
\texttt{ddprocess.sh} is a bash (Shell) script to process a network in network mode, using the double-difference approach.\\
\textbf{Location:} \texttt{src/bash/ddprocess.sh}

\subsubsection{Usage}
\texttt{ddprocess -y YYYY -d DDD [OPTIONS]}\\

Switches:
\begin{itemize}
\item \texttt{-a --analysis-center=}[option]\\
Specify the analysis center. This can be either
\begin{enumerate}
\item igs, or
\item cod
\end{enumerate}
\underline{Default value: cod}.
\item \texttt{-b --bernese-loadvar=} /foo/bar/LOADGPS.setvar\\
Specify the Bernese LOADGPS.setvar file; this is needed to resolve the Bernese-related variables.
\item \texttt{-c --campaign=}[option]\\
Specify the campaign name. This name should be exactly the name of the campaign used within
Bernese. Inside the script it is truncated to upper-case, so the options
\texttt{--campaign=greece} and \texttt{--campaign=GREECE} are equivelant. For a list of all the files
depending on this variable, see \autoref{sec:ddprocess_notes}, Note 1.
Only specify the name of the campaign do \textbf{NOT} include the path.
\item \texttt{-d --doy=}[option]\\
Specify the day of year (as integer).
\item \texttt{-e --elevation-angle=}\\
Specify the elevation cut-off angle. The angle is expressed as integer in degrees.
\underline{Default value: 3}.
\item \texttt{-f --ion-products=}[option]\\
Specify (a-priori) ionospheric correction file identifier. If more than one, use a comma-seperated list 
(e.g. \texttt{-f FFG,RFG}). See \autoref{sec:ddprocess_notes}, Note 2.
\item \texttt{-i --solution-id=}[option]\\
Specify solution id (e.g. \texttt{FFG}). See \autoref{sec:ddprocess_notes}, Note 3.
\item \texttt{-l --stations-per-cluster=}[option]\\
Specify the number of stations per cluster. Input should be apositive integer.
\underline{Default value is 5}.
\item \texttt{-m --calibration-model=}[option]\\
The extension (model) used for antenna calibration files. This can be e.g. \texttt{I01, I05 or I08}. 
What you enter here, will be appended to the pcv filename (provided via the \texttt{-p} switch) and all 
calibration-dependent Bernese processing files (e.g. \texttt{SATELLITE.XXX}). 
See \autoref{sec:ddprocess_notes}, Note 4.
\item \texttt{-p --pcv-file=}[option]\\
Specify the PCV file to be used. Do not provide the extension (this is
automatically appended using the \texttt{-m} switch). See \autoref{sec:ddprocess_notes}, Note 4.
\item \texttt{-r --save-dir=}[option]\\
Specify directory where the solution will be saved; note that if the directory does not exist, 
it will be created.
\item \texttt{-s --satellite-system=}[option]\\
Specify the satellite system; this can be:
\begin{enumerate}
\item gps, or
\item mixed (i.e. gps+glonass)
\end{enumerate}
\underline{Default value is gps}.
\item \texttt{-t --solution-type=}[option]\\
Specify the solution type; this can be:
\begin{enumerate}
\item final, or
\item urapid
\end{enumerate}
\item \texttt{-u --update=}[option]\\
Specify which records/files should be updated; valid values are:
\begin{enumerate}
\item \texttt{crd} : update the default network crd file.
\item \texttt{sta} : update station-specific files, i.e. time-series records for the stations.
\item \texttt{ntw} : update update network-specific records.
\item \texttt{all} : all both the above.
\end{enumerate}
More than one options can be provided, in a comma seperated string e.g.
\texttt{--update=crd,sta}.
\item \texttt{-y --year=}[option]\\
Specify the year as a 4-digit integer.
\item \texttt{-x --xml-output}\\
Produce an xml (actually docbook) output summary report.
\item \texttt{--force-remove-previous}\\
Remove any files from the specified save directory (\texttt{-r --save-dir=}) prior to start 
of processing.
\item \texttt{--add-suffix=}[option]\\
Add a suffix (e.g. \texttt{\_GPS}) to saved products of the processing.
\item \texttt{-h --help}\\
Display (this) help message and exit.
\item \texttt{-v --version}\\
Dsiplay version and exit.
\end{itemize}


\subsubsection{Prerequisites}


\subsubsection{Exit Status}
On sucess, the program returns \texttt{0}.

\subsubsection{ToDo}
%\begin{tabular}{l l l}
%Date & What & Status\\
%\hline \\
%\end{tabular}

\subsubsection{Bugs}
Send reports to:\\
Xanthos Papanikolaou \href{mailto:xanthos@mail.ntua.gr}{mailto:xanthos@mail.ntua.gr}\\
Demitris Anastasiou  \href{mailto:danast@mail.ntua.gr}{mailto:danast@mail.ntua.gr}\\
Vangelis Zacharis  \href{mailto:vanzach@survey.ntua.gr}{mailto:vanzach@survey.ntua.gr}\\
\bigskip

%\begin{tabular}{l l l}
%Date & What & Status\\
%\hline \\
%\end{tabular}

\subsubsection{Notes}\label{sec:ddprocess_notes}
\begin{enumerate}[label=\arabic*]
\item A list of files are expected to be present in the tables directory, specified by
the campaign name. See \autoref{tab:ddprocess_notes}:

\begin{tabular}{l | l}\label{tab:ddprocess_notes}
Expected File & Linked to\\
\hline
 \$\{TABLES\}/pcv/\$\{PCV\_FILE\}       & \$\{X\}/GEN/\$\{PCV\_FILE\} \\
 \$\{TABLES\}/sta/\$\{CAMPAIGN\}.STA    & \$\{P\}/STA/\$\{CAMPAIGN\}.STA\\
 \$\{TABLES\}/blq/\$\{CAMPAIGN\}.BLQ    & \$\{P\}/STA/\$\{CAMPAIGN\}.BLQ\\
 \$\{TABLES\}/atl/\$\{CAMPAIGN\}.ATL    & \$\{P\}/STA/\$\{CAMPAIGN\}.ATL\\
 \$\{TABLES\}/crd/\$\{CAMPAIGN\}.igs    & \\
 \$\{TABLES\}/crd/\$\{CAMPAIGN\}.epn    & \\
 \$\{TABLES\}/crd/\$\{CAMPAIGN\}.reg    & \\
\hline
\end{tabular}
\item The ionospheric correction file, must be in the Bernese-specific ION format.
These files should reside in the product area, specified by the variable \$\{PRODUCT\_AREA\}
stored as \$\{PRODUCT\_AREA\}/YYYY/DDD/XXXYYDDD0.ION.Z, where \texttt{XXX} is the solution identifier
specified by the \texttt{-f} option.\\
If none of these files are found (or if the \texttt{-f} switch is not used), then the script
will try to download a Bernese-specific ION file from CODE's ftp, using the program
wgetion. This downloaded files can be final, rapid or ultra-rapid.
\item The solution id will have an effect on the naming of the Final, Preliminary
and Size-Reduced solution files. If e.g. the solution-id is set to \texttt{NTA}, then
the Final solution files will be named \texttt{NTA}, the preliminary \texttt{NTP} 
and the size-reduced \texttt{NTR}.
\item The pcv file must reside in the \texttt{tables/pcv} folder, and will be linked by the
script to the \{\%GEN\} directory. Do not provide the extension; it will be automatically
generated using the pcv file and the extension given via the calibration model (\texttt{-m}).
E.g. using \texttt{-p GRE\_PCV} and \texttt{-m I08}, then the script will search for the pcv file
\$\{TABLES\}/pcv/GRE\_PCV.I08.
\end{enumerate}