\subsection{Protocol Control File ntua\_ddp.pcf}
\label{ntuaddppcf}

\subsubsection{Copy required files \& Coordinates}\label{ddp-crfc}
\begin{verbatim}
3** 8******* 8******* 8******* 8******* 1 3** 3** 3** 3** 3** 3** 3** 3** 3** 3**
#
# Copy required files
# -------------------
## 001 NDP_COP  NDP_GEN           ANY      1
## 002 ATX2PCV  NDP_GEN           ANY      1 001
## 003 COOVEL   NDP_GEN           ANY      1 001
004 COOVEL   NDP_GE2           ANY      1
005 CRDMERGE NDP_GEN           ANY      1 004
## 011 RNX_COP  NDP_GEN           ANY      1 001
## 021 OBSMRGAP NDP_GEN           ANY      1 011
## 022 OBSMRG_P NDP_GEN           ANY      1 021
## 031 ION_MRG  NDP_GEN           ANY      1 011
## 099 DUMMY    NDP_GEN           ANY      1 002 005 022 031
\end{verbatim}

\begin{itemize}
  \item \texttt{COOVEL (NDP\_GE2)} \underline{Extrapolate Coordinates - COOVEL}
    Give the input coordinate file \texttt{\$(REFINF)\_R.CRD} and the velocity
    file \texttt{\$(REFINF)\_R.VEL} extrapolate the coordinates (to the epoch
    currently set) to the coordinate file \texttt{REF\$YSS+0.CRD}.
  \item \texttt{CRDMERGE (NDP\_GEN)} \underline{Merge Coordinate/Velocity Files - CRDMERGE}
    Given the coordinate file \texttt{REF\$YSS+0.CRD} and the (coordinate) file
    \texttt{REG\$YSS+0.CRD}, merge the coordinates to the resulting file
    \texttt{\$A\$YSS+0.CRD}. Will also produce the file \texttt{REF\$YSS+0.FIX}.
    Note that the default flag priority is used, i.e. the sequence: 
    \texttt{\{R, C, U, T, P, M, A, W, N, I\}}. Flag \texttt{I} is used for the
    creation of the station selection file.
\end{itemize}

\subsubsection{Prepare Orbit \& Pole}\label{ddp-poap}
\begin{verbatim}
# Prepare the orbits
# ------------------
101 POLUPDH  NDP_GEN           ANY      1 005
## 111 ORBMRGH  NDP_GEN           ANY      1 101
## 112 PRETAB   NDP_GEN           ANY      1 111
112 PRETAB   NDP_GEN           ANY      1 101
113 ORBGENH  NDP_GEN           ANY      1 112
## 199 DUMMY    NO_OPT            ANY      1 113
\end{verbatim}

\begin{itemize}
  
  \item \texttt{POLUPDH (NDP\_GEN)} \underline{Convert IERS pole to Bernese format - POLUPDH}
    \begin{bclogo}[logo=\bcattention, couleurBarre=red, noborder=true, couleur=Peach]{Important!}
      Depending on the Analysis Center, this input panel should hold different
      value, or better, different text edit windows should be filled.
    \end{bclogo}
    Given an input \texttt{erp} file this program will convert the file to a
    Bernse-formated \texttt{erp} file \texttt{\$B\$YSS+0.ERP}. In case a CODE
    \texttt{erp} file is used, then the program expects as input a file named
    \texttt{\$B\$WD+0.ERP}. In case a foreign-formated \texttt{erp} file is
    given, it is expected as \texttt{\$B\$WD+0.IEP}.
  
  \item \texttt{PRETAB (NDP\_GEN)} \underline{Create Tabular Orbits - PRETAB}
    Given the input precise ephemeris file \texttt{\$B\$WD+0.SP3}, the pole file
    \texttt{\$B\$YSS+0.ERP}, ocean loading corrections \texttt{\$(BLQINF).BLQ}
    and atmospheric loading corrections \texttt{\$(ATLINF).ATL}, produce the
    tabular file \texttt{\$B\$YSS+0.TAB} and the satellite clock file 
    \texttt{\$B\$YSS+0.CLK}. Also needs the satellite problem file \texttt{\$(SATCRX).CRX}.


    \textcolor{Cerulean}{Note that the files \texttt{\$(ATLINF).ATL} and 
    \texttt{\$(BLQINF).BLQ} are needed for Center of Mass Corrections. In PANEL
    \texttt{PRETAB 2: General Options}, there is an option 
    \texttt{Apply CMC correction} for \texttt{BLQ} and/or \texttt{ATL}.}

  \item \texttt{ORBGENH (NDP\_GEN)} \underline{Create Standard Orbits - ORBGEN}
    Given the tabular orbit \texttt{\$B\$YSS+0.TAB} the pole file \texttt{\$B\$YSS+0.ERP},
    and ocean and atmospheric loading corrections, \texttt{\$(BLQINF).BLQ} and
    \texttt{\$(ATLINF).ATL}, this program will produce the standard orbit file:
    \texttt{\$B\$YSS+0.STD}. The output summary file \texttt{ORB\$YSS+0.LST} is
    created.

    General input files include, the satellite problem file \texttt{\$(SATCRX).CRX},
    and the satellite inormation file \texttt{\$(SATINF).\$(PCV)}.

    See note above (PRETAB) for loading corrections; the same holds here and should
    be consistent.
\end{itemize}

\subsubsection{Preprocess}\label{ddp-prepro}
\begin{verbatim}
# Preprocess, convert, and synchronize observation data
# -----------------------------------------------------
201 RNXGRA   NDP_GEN           ANY      1 004
## 211 RNXSMTAP NDP_GEN           ANY      1 201
## 212 RNXSMT_H NDP_GEN           ANY      1 211
221 RXOBV3AP NDP_GEN           ANY      1 201
222 RXOBV3_H NDP_GEN           ANY      1 221
231 CODSPPAP NDP_GEN           ANY      1 113 222
232 CODSPP_P NDP_GEN           ANY      1 231
233 CODXTR   NDP_GEN           ANY      1 232
## 299 DUMMY    NO_OPT            ANY      1 233
\end{verbatim}
\begin{itemize}
  \item \texttt{RNXGRA (NDP\_GEN)} \underline{Create Observation Statistics - RNXGRA}
  The program input is all RINEX files \texttt{????\$S+0.\$YY+0O}. It outputs
  the summary file \texttt{GRA\$YSS+0.SMC}. Needs the station information file
  \texttt{\$(CRDINF).STA}. \textcolor{Cerulean}{W T F ? ? Giati to exoume auto ??}

  \item \texttt{RXOBV3AP (NDP\_GEN)} \underline{Rinex to Bernese format - RXOBV3}
    The program input is all RINEX files \texttt{????\$S+0.\$YY+0O} and the
    station information file \texttt{\$(CRDINF).STA}. 

    Other required input files include, the satellite problem file 
    \texttt{\$(SATCRX).CRX}, the satellite inormation file \texttt{\$(SATINF).\$(PCV)},
    the phase center offset file \texttt{\$(PCVINF).\$(PCV)}, and the abbreviation
    file \texttt{\$(CRDINF).ABB}.

    Note that, this program uses also the variable: \texttt{\$SATSYS} (which
    satellite system to be extracted from RINEX).

    \textcolor{OrangeRed}{Open Issues:}
    \begin{enumerate}
      \item gather station names from \texttt{MARKER NAME} or \texttt{MARKER DOME}?
      \item if station not in abbrevioation list : \texttt{UPDATE} or \texttt{ERROR} ?
      \item sampling interval is set to 30 seconds; maybe make this a variable,
        so that we can handle differently different occasions, e.g. hourly
        processing.
      \item There is a minimum required number of epochs for each RINEX (set to
        10 epochs). Is this ok with both 24hour and 1hour processing?
      \item handling of known inconsistencies -> \texttt{\$(CRXINF)} ??
    \end{enumerate}

  \item \texttt{CODSPPAP (NDP\_GEN)} \underline{Code Based Clock Synchronization - CODSPP}
    Given a standard orbit file \texttt{\$B\$YSS+0.STD}, the satellite clock file
    \texttt{\$B\$YSS+0.CLK}, the code observation files of type \texttt{????\$S+0.CZH},
    the pole file \texttt{\$B\$YSS+0.ERP}, the dcb file \texttt{P1C1\$M+0.DCB} and
    the a-priori coordinate file \texttt{\$A\$YSS+0.CRD}, estimate the receiver
    clock corrections.

    No station information file used; we need  the satellite problem file 
    \texttt{\$(SATCRX).CRX}, the satellite inormation file \texttt{\$(SATINF).\$(PCV)},
    the phase center offset file \texttt{\$(PCVINF).\$(PCV)}.

    Runs in cluster mode, no coordinates estimates, troposphere modelling = GMF.
    Minimun elevation = 10$^{\circ}$.
    
    \textcolor{OrangeRed}{Open Issues:}
    \begin{enumerate}
      \item save residuals (ascii/zipped) ??
      \item develop a program to plot clock RINEX
    \end{enumerate}

  \item \texttt{CODXTR (NDP\_GEN)} \underline{Extract CODSPP Output - CODXTR}
    Given the output file(s) \texttt{SPP\$S+0???.OUT}, it produces the summary
    file \texttt{SPP\$YSS+0.OUT}.

    \textcolor{OrangeRed}{Open Issues:}
    \begin{enumerate}
      \item program to plot the output file (already have one ??)
      \item program to convert the output file to html
    \end{enumerate}

\end{itemize}
