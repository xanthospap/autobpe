\subsection{updatecrd}
\label{updatecrd}

\subsubsection{Purpose}
\texttt{updatecrd.py} is a Python script used to update the (cartesian) coordinates in a
Bernese-formated \texttt{.CRD} file, using a second \texttt{.CRD} file as reference.\\
\textbf{Location:} \texttt{src/python/updatecrd.py}

\subsubsection{Usage}
\texttt{updatecrd -r [.CRD-file] -u [.CRD-file] [OPTIONS]}\\
at least \texttt{--update-file=} and \texttt{--reference-file=}
must be specified as command line arguments.\\

Switches:
\begin{itemize}
\item \texttt{-u --update-file=} [.CRD-file]\\
The \texttt{.CRD} file to be updated.
\item \texttt{-r --reference-file=} [.CRD-file]\\
The \texttt{.CRD} file to be used as reference.
\item \texttt{-s --stations=} [station1,station2]\\
A comma-seperated list of stations, whose coordinates should be updated. 
If not specified, then all stations found in the file-to-be-updated will be
updated, if they are macthed in the reference .CRD file.
\item \texttt{-f --flags=} [A,W]\\
Only update stations flaged with specific characters in the 
reference file. The flags should be specified using a comma-seperated list, 
e.g. \texttt{--flags=W,A}. If this switch is not specified, 
all matching stations will be updated regardless of their flags.
\item \texttt{--include-unmatched}\\
The final output CRD file, will contain all the stations recorded
in the reference file (either matched or not).
\item \texttt{--delete-unmatched}\\
By default, all stations recorded in the file-to-be-updated will be included
in the resulting file. If this (\texttt{--delete-unmatched}) switched is turned on,
then stations that are not matched in the reference file will not be included in the
resulting file.
\item \texttt{--no-marker-number}\\
If specified, the marker number will not be used when trying to match
stations (i.e. \texttt{YEBE 13420M001} will be matched to \texttt{YEBE}).
\item \texttt{-h --help}\\
Display (this) help message and exit.
\item \texttt{-v --version}\\
Dsiplay version and exit.
\end{itemize}

\subsubsection{Prerequisites}
\begin{itemize}
\item \texttt{bernutils.berncrd} Python (library) module.
\end{itemize}

\subsubsection{Exit Status}
On sucess, the program returns \texttt{0}.\\
Else, the return status is $>$0.

\subsubsection{ToDo}
%\begin{tabular}{l l l}
%Date & What & Status\\
%\hline \\
%\end{tabular}

\subsubsection{Bugs}
Send reports to:\\
Xanthos Papanikolaou \href{mailto:xanthos@mail.ntua.gr}{mailto:xanthos@mail.ntua.gr}\\
Demitris Anastasiou  \href{mailto:danast@mail.ntua.gr}{mailto:danast@mail.ntua.gr}\\
Vangelis Zacharis  \href{mailto:vanzach@survey.ntua.gr}{mailto:vanzach@survey.ntua.gr}\\
\bigskip

\begin{tabular}{l l l}
Date & What & Status\\
\hline \\
JUN, 2015 & help and version switch not working. Fix & Waiting ...\\
\end{tabular}