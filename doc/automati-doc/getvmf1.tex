\subsection{getvmf1}
\label{getvmf1}

\subsubsection{Purpose}
\texttt{getvmf1.py} is a Python script to download VMF1 (i.e. Vienna Mapping Function 1)
grid files. These files are located on the web at \url{http://ggosatm.hg.tuwien.ac.at}.
Grid files older than today are always placed in 
\url{http://ggosatm.hg.tuwien.ac.at/DELAY/GRID/VMFG/YYYY} while prediction files (including
today and forward) are placed in \url{http://ggosatm.hg.tuwien.ac.at/DELAY/GRID/VMFG_FC/YYYY}.

The grid files hold values per 6 hours (so we have 4 files per day), at 00, 06, 12 and 18
hours and they are named as \texttt{VMFG\_YYYYMMDD.HSS}, where:
\begin{itemize}
    \item \texttt{YYYY} is the year,
    \item \texttt{MM} is the month,
    \item \texttt{DD} is the day of month and
    \item \texttt{SS} is the hour (00, 06, 12 or 18)
\end{itemize}

Note that the grid files previous than 01/01/2009 are compressed in \texttt{.gz} format;
the routine will automatically uncompress these files.

\subsubsection{Usage}
\texttt{getvmf1 -y [year] -d [doy] [OPTIONS]}\\
at least \texttt{--year=} and \texttt{--doy=}
must be specified as command line arguments.\\

Switches:
\begin{itemize}
\item \texttt{-y --year=}[YYYY]\\
The \texttt{YEAR} for which the grid file is needed.
\item \texttt{-d --doy=}[Day Of Year]\\
The \texttt{day of year} for which the grid file is needed.
\item \texttt{-r --hour=}[Hour]\\
If we only want a single grid file to cover a certain time interval, then this switch
can be used. The \texttt{hour} argument should be a valid integer of float in the range
[0,24).
\item \texttt{-o --outdir=}[output directory]\\
Specify the directory where the downloaded file(s) will be saved.
\item \texttt{-h --help}\\
Display (this) help message and exit.
\item \texttt{-v --version}\\
Dsiplay version and exit.
\end{itemize}

\subsubsection{Prerequisites}
\begin{itemize}
\item \texttt{bernutils.berncrd} Python (library) module.
\end{itemize}

\subsubsection{Exit Status}
On sucess, the program returns \texttt{0}.\\
Else, the return status is $>$0.\\


The routine will also report the list of downloaded and saved files to \texttt{stdout}, e.g.
\begin{Verbatim}[fontsize=\scriptsize]
$> getvmf1 -y 2007 -d 10 -o /foo/bar/data
Downloaded http://ggosatm.hg.tuwien.ac.at/DELAY/GRID/VMFG/2007/VMFG_20070110.H00.gz to /foo/bar/data/VMFG_20070110.H00
Downloaded http://ggosatm.hg.tuwien.ac.at/DELAY/GRID/VMFG/2007/VMFG_20070110.H06.gz to /foo/bar/data/VMFG_20070110.H06
Downloaded http://ggosatm.hg.tuwien.ac.at/DELAY/GRID/VMFG/2007/VMFG_20070110.H12.gz to /foo/bar/data/VMFG_20070110.H12
Downloaded http://ggosatm.hg.tuwien.ac.at/DELAY/GRID/VMFG/2007/VMFG_20070110.H18.gz to /foo/bar/data/VMFG_20070110.H18
\end{Verbatim}

\subsubsection{ToDo}
\begin{tabular}{l l l}
Date & What & Status\\
\hline \\
20,Jul,15  & add switch for help and version & Waiting ...\\
\end{tabular}

\subsubsection{Bugs}
Send reports to:\\
Xanthos Papanikolaou \href{mailto:xanthos@mail.ntua.gr}{mailto:xanthos@mail.ntua.gr}\\
Demitris Anastasiou  \href{mailto:danast@mail.ntua.gr}{mailto:danast@mail.ntua.gr}\\
Vangelis Zacharis  \href{mailto:vanzach@survey.ntua.gr}{mailto:vanzach@survey.ntua.gr}\\
\bigskip

\begin{tabular}{l l l}
Date & What & Status\\
\hline \\
JUN, 2015 & help and version switch not working. Fix & Waiting ...\\
\end{tabular}