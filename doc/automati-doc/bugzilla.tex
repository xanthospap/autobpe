\subsection{Project's Bugzilla}\label{subsec:bugs1}
There is a dedicated bugzilla server for the project located at \url{http://dionysos.survey.ntua.gr/bugzilla/}.
Bugs related to the scripts/programs described in this document, are found under
\texttt{Product} \textbf{automati}, \texttt{Component} \textbf{bin}. Other relevant
components include \texttt{bernutils}, \texttt{databases} and \texttt{documentation}.

To report or scan bugs, the user must have a username/password.

\subsubsection{Reporting Bugs}
If you do not have a valid username/password to connect to the bugzilla host, ask
\href{mailto:danast@mail.ntua.gr}{Mitsos} to create one.

If you want to report a bug, follow these set of rules:
\begin{enumerate}
\item always report using the \texttt{Product} \textbf{automati},
\item report script bugs using the \texttt{Component} \textbf{bin}. \textbf{Do not use the
other components except is you are absolutely certain that the bug is caused by a specific
component}.
\item include as many information as possible; it is important that you include \textbf{how}
you run the program, i.e. exactly which command you issued.
\end{enumerate}

\subsubsection{Fixing Bugs}
Do not use the \texttt{master} branch to fix bugs. Create a new branch, named after
the bug; fix the bug and merge. Always mark the bug as \texttt{RESOLVED}.

\subsection{Report to Developers}
Users can also directly report bugs, to:
Xanthos Papanikolaou \href{mailto:xanthos@mail.ntua.gr}{mailto:xanthos@mail.ntua.gr},
Demitris Anastasiou  \href{mailto:danast@mail.ntua.gr}{mailto:danast@mail.ntua.gr}, or
Vangelis Zacharis  \href{mailto:vanzach@survey.ntua.gr}{mailto:vanzach@survey.ntua.gr}