\subsection{adnq2html}
\label{adnq2html}

\subsubsection{Purpose}
\texttt{adnq2html.py} is a Python script to collect information off from a Bernese v5.2
\texttt{ADDNEQ2} output/summary file and translate it to html format. The result
should be a self-explanatory, complete html file/page.

The script will read various records and report a lot of information, including
reference frame, solution statistics, general information, station-specific coordinates,
etc. The bigest part of the output, is a table holding station entries (e.g. coordinates,
coordinate corrections, etc).

\subsubsection{Usage}
\texttt{adnq2html -f [ADDNEQ2 FILE] [OPTIONS]}\\
at least \texttt{-f} must be specified as command line argument.\\

Switches:
\begin{itemize}
\item \texttt{-f --addneq-file=}[ADDNEQ2 FILE]\\
A valid ADDNEQ2 output file, to extract information from.
\item \texttt{-t --table-entries=}[FORMAT\_STRING]\\
This switch (if used) controlls what station-specific information are going to be
translated and recorded to the output (html) table. The format of this option,
(i.e. the \texttt{FORMAT\_STRING}) is very strict; a collection of comma-seperated
substrings is expected, each of which describes a field to appear on the table 
(in that specific order). The names of the valid substrings is listed in 
\autoref{tab:format_str}. 
\underline{Default value is 'latcor,dn,loncor,de,hgtcor,du'}.
\textbf{example} : \texttt{adnq2html [...] -t latcor,dn,loncor,de,hgtcor,du} will
create (among other html output) a table with 7 columns, where the first column
will be the station name, the second the correction (i.e. a-priori $-$ a-posteriori)
for the Latitude component, the third the correction for Longtitude, the forth
the correction for Height; the fifth column will hold the dNorth (toppocentric),
the sixth the dEast and the last the dUp.
\item \texttt{-w --warnings-str=}[FORMAT\_STRING]\\
This switch (if used) will trigger warning messages to appear, if a station record
is outside a given limit. \textit{Of-course, this is only helpful for station-specific 
fields like coordinate corections, or coordinate sigmas, etc}. The format of this option,
(i.e. the \texttt{FORMAT\_STRING}) is very strict; it should be created from 
comma-seperated substrings, each of which describes a field to be checked against 
a given value. These sub-strings are made-up of thwo parts:
\begin{enumerate}
\item the name of a valid field to check, and
\item limit to check against
\end{enumerate}
The two parts are seperated by a \texttt{'='} char. The names of the valid fields 
to check are listed in \autoref{tab:format_str}.
\textbf{example} : \texttt{adnq2html [...] -w dn=.001,de=.002,du=.003} will produce a
warning message, for each station if the dNorth (for the vector a-priori $-$ a-posteriori)
is larger in \textbf{absolute value} than .001 meters, or if dEast is larger than
.002 meters of if the dUp is larger than .003 meters.
\item \texttt{-h --help}\\
Display (this) help message and exit.
\item \texttt{-v --version}\\
Dsiplay version and exit.
\end{itemize}

\subsubsection{Prerequisites}
\begin{itemize}
\item \texttt{bernutils.badnq} Python (library) module.
\item If enabled, the warning messages emmited from the script use the 
\href{http://getbootstrap.com/2.3.2/index.html}{Bootstrap framework}
to produce nice-looking warning boxes. Nothing is needed to do this actually as all required
elements are just written in the html header, and accessed directly from the browser when
the file is opened.
\end{itemize}

\subsubsection{Exit Status}
On sucess, the program returns \texttt{0}.\\
Else, the return status is $>$0.

\subsubsection{ToDo}
\begin{tabular}{l l l}
Date & What & Status\\
\hline \\
SEP, 2015 & We should produce warnings for more entries, not just station-specific info. E.g. a-posteriori rms & \\
\end{tabular}

\subsubsection{Bugs}
See \autoref{subsec:bugs1}.

\subsubsection{Format Strings}

\begin{tabular}{l l}\label{tab:format_str}
String & Meaning \\
\hline\\
\texttt{x}   & X-component (estimated)\\
\texttt{xapr}& X-component (a-priori value)\\
\texttt{xcor}& X-component (estimated $-$ a-priori)\\
\texttt{xrms}& X-component (rms)\\
\hline\\
\texttt{y}   & Y-component (estimated)\\
\texttt{yapr}& Y-component (a-priori value)\\
\texttt{ycor}& Y-component (estimated $-$ a-priori)\\
\texttt{yrms}& Y-component (rms)\\
\hline\\
\texttt{z}   & Z-component (estimated)\\
\texttt{zapr}& Z-component (a-priori value)\\
\texttt{zcor}& Z-component (estimated $-$ a-priori)\\
\texttt{zrms}& Z-component (rms)\\
\hline\\
\texttt{lat}   & (ellipsoidal) latitude (estimated)\\
\texttt{latapr}& latitude (a-priori value)\\
\texttt{latcor}& latitude (estimated $-$ a-priori)\\
\texttt{latrms}& latitude (rms)\\
\hline\\
\texttt{lon}   & (ellipsoidal) longtitude (estimated)\\
\texttt{lonapr}& longtitude (a-priori value)\\
\texttt{loncor}& longtitude (estimated $-$ a-priori)\\
\texttt{lonrms}& longtitude (rms)\\
\hline\\
\texttt{hgt}   & (ellipsoidal) height (estimated)\\
\texttt{hgtapr}& height (a-priori value)\\
\texttt{hgtcor}& height (estimated $-$ a-priori)\\
\texttt{hgtrms}& height (rms)\\
\hline\\
\texttt{adj}   & Adjustment option (i.e. Fixed, Free, ...)\\
\hline\\
\texttt{dn}    & North (estimated $-$ a-priori)\\
\texttt{de}    & East (estimated $-$ a-priori)\\
\texttt{du}    & Up (estimated $-$ a-priori)\\
\hline
\end{tabular}