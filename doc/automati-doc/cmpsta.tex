\subsection{cmpsta}
\label{cmpsta}

\subsubsection{Purpose}
\texttt{cmpsta.py} is a Python script to compare two Bernese v5.2 station information
files (default extansion \texttt{.STA}). Station information files are divided into
blocks of information, called \textbf{Types}; the most important are:
\begin{itemize}
\item the \texttt{Type 001} block, in which the station name or renaming are recorded
(normally using UNIX wildcards), and
\item the \texttt{Type 002} block, in which the equipment information are recorded.
\end{itemize}
The script only reads and compares the two first blocks, i.e. \texttt{Type 001} and \texttt{Type 002}.


Note that the format of these station information files is very strict and the script
can only run successefuly if the input files follow this format. This goes for all
stations recorded in the files, \textbf{not} only the ones to be read/compared.


Note also the input and output stations names, correspond to the column \texttt{OLD STATION NAME}
in the \texttt{Type 001} block, and the matching is performed using UNIX shell-like
wildcard notation.

For example, given the following extract of a station information file:
\begin{Verbatim}[fontsize=\scriptsize]
STATION NAME          FLG          FROM                   TO         OLD STATION NAME      REMARK
****************      ***  YYYY MM DD HH MM SS  YYYY MM DD HH MM SS  ********************  ************************
AIRA 21742S001        001  1980 01 06 00 00 00  2099 12 31 00 00 00  AIRA*                 MGEX,aira_20120821.log
AUT0 49431M001        001                                            AUT0*                 MGEX,aut0_20121227.log
AXPV 10057M001        001                                            AXPV*                 MGEX,axpv_20130515.log
\end{Verbatim}
if we start the script, comparing the station \texttt{'AIRA'}, then this will be matched
to \texttt{'AIRA 21742S001'}.

\subsubsection{Usage}
\texttt{cmpsta -l [STA1] -r [STA2] [OPTIONS]}\\
at least \texttt{-l} and \texttt{-r}
must be specified as command line arguments.\\

Switches:
\begin{itemize}
\item \texttt{-l --local-sta=}[STAFILE]\\
The station information file to compare. This must be a valid .STA file (including path).
\item \texttt{-r --reference-sta=}[STAFILE]\\
The station information file to compare against. If this file is not found, then the 
script will try to download it from CODE's ftp server, at \url{ftp://ftp.unibe.ch//aiub/BSWUSER52/STA/}
\item \texttt{-s --stations=}[STA1,STA2,...]\\
A comma-seperated list of stations to compare. These are going to be matched using the 
\texttt{OLD STATION NAME} column(s) in the \texttt{Type 001} block. If no stations are
given, then the script will compare all stations.
\item \texttt{--splitout}\\
By default, all discrepancies (if found) are reported to \texttt{stdout}. If this switch is on,
then the discrepancies will be recorded to station-specific files, named as 
\texttt{\$(station).stadf}\footnotemark[1].
\item \texttt{--no-marker-numbers}\\
By default, station names are mathced using both the 4-char id and the station number (if any).
Hence, station \texttt{'WTZR'} will not be considered the same station as \texttt{'WTZR 14201M010'}.
If this switch is enabled, then when comparing station names only the first four 
chars are considered (which should be the station id), hence in the previous example
station names \texttt{'WTZR'} and \texttt{'WTZR 14201M010'} would be considered equal.
\textbf{Use with care}, sometimes station names are necessary to distinguish between stations!
\item \texttt{-h --help}\\
Display (this) help message and exit.
\item \texttt{-v --version}\\
Dsiplay version and exit.
\end{itemize}

\subsubsection{Prerequisites}
\begin{itemize}
\item \texttt{bernutils} Python (library) module.
\end{itemize}

\subsubsection{Exit Status}
\begin{tabular}{l l}
\label{tab:cs_exit_status}
Exit Code & Status\\
\hline
0 & Success; no discrepancies between the files \\
1 & Error (most probably format error) \\
2 & Missing stations(s) from compared .STA file\footnotemark[2]\\
3 & Discrepancies for one (or more) stations\\
\hline
\end{tabular}

\subsubsection{ToDo}
\begin{tabular}{l l l}
Date & What & Status\\
\hline \\
15,Sep,15  & add switch for help and version & Waiting ...\\
\end{tabular}

\subsubsection{Bugs}
%Send reports to:\\
%Xanthos Papanikolaou \href{mailto:xanthos@mail.ntua.gr}{mailto:xanthos@mail.ntua.gr}\\
%Demitris Anastasiou  \href{mailto:danast@mail.ntua.gr}{mailto:danast@mail.ntua.gr}\\
%Vangelis Zacharis  \href{mailto:vanzach@survey.ntua.gr}{mailto:vanzach@survey.ntua.gr}\\
See \autoref{subsec:bugs1}.
\bigskip

\begin{tabular}{l l l}
Date & What & Status\\
\hline \\
SEP, 2015 & help and version switch not working. Fix & Waiting ...\\
\end{tabular}

\footnotetext[1]{If the station name contains wildcard characters (i.e. '?' or '*')
then these will not be used to form the filename for the discrepancies. '?' will be
substituted with '\_' and '*' with '\_\_'}
\footnotetext[2]{Missing stations are only reported if the occur in the compared .STA file 
(i.e. the one specified with the \texttt{-l} option). If a station is missing from the reference
.STA file, then the station is just skipped and no error is reported (only a warning message
is printed).}