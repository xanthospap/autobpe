\subsection{rnxdwnl}
\label{rnxdwnl}

\subsubsection{Purpose}
\texttt{rnxdwnl.py} is a Python script to download RINEX files.\\
\textbf{Location:} \texttt{src/python/rnxdwnl.py}

\subsubsection{Usage}
\texttt{rnxdwnl -y YYYY -d DDD [OPTIONS]}\\

Switches:
\begin{itemize}
\item \texttt{-s, --stations=} station1,stations2,...\\
Specify a comma-seperated list of stations to be downloaded. The stations specified here, 
must be included in the database. The name of stations specified, must be the one
used by DSO (a 4-char string). In some, rare cases, this may not match the 'official'
4-char name of the station. When the rinex file is searched for on the web, the
'official' name will be used.
\item \texttt{-n, --networks=} network1,network2,...\\
Specify a comma-seperated list of networks. Every station, belonging to these
networks will (try to) be downloaded. The name of the network(s) specified, must 
match valid networks in the database.
\item \texttt{-y, --year=} YYYY\\
Specify the year for which the RINEX are requested. This must be a (valid),
4-digit integer.
\item \texttt{-d, --doy=} DDD\\
Specify the day of year for which the RINEX are requested. This must be a (valid), integer.
\item \texttt{-f, --force-remove}\\
By default, if the RINEX to be downloaded already exists, with size $>$ 0, then
the downloading step is not performed (other steps -if specified- are perfored normaly).
If this switch is turned on, then if a file exists,
named exactly as the one to be downloaded, then this file is removed and a normal,
download is performed.
\item \texttt{-u, --uppercase}\\
If turned on, the downloaded RINEX file(s) will be truncated to upper-case.
\item \texttt{-p, --path=} /foo/bar\\
Specify the directory (path) where the downloaded files will be downloaded to.
\item \texttt{-z, --uncompress}\\
Uncompress the downloaded RINEX file(s). Note that this will only work for UNIX compressed '.Z' files.
\end{itemize}

\subsubsection{Prerequisites}
The Python library MySQLdb must be installed and available for importing.\\

The program needs to connect to a (MySQL) database called \index{procsta}\texttt{procsta}, 
and make various queries
about the stations, networks, servers, etc. The structure of this database
is strict, cause the queries are hardcoded in the source code of this program.\\

For more information on the used database, ask \href{mailto:danast@mail.ntua.gr}{Mitsos}.\\

The program is designed to work on UNIX-like systems. It will call the Shell to issue commands (like
downloading, compressing, etc). Depending on the specific use, the following may be required:
\begin{itemize}
\item \texttt{compress} and/or \texttt{uncompress} utilities; needed to compress or uncompress
UNIX-compressed files (i.e. \texttt{.Z}).
\item \texttt{wget} needed to download remote files, when the protocol is \texttt{http}, \texttt{ftp}, or \texttt{https}.
\item \texttt{scp} needed to download remote files, when the protocol is \texttt{ssh}. Note that if a non-standard port
is used for \texttt{ssh}, then it must be hardcoded into the source code. Also, the remote and server sites, must be able to
connect without using explicit passwords (they must hold the ssh keys).
\end{itemize}

\subsubsection{Exit Status}
On sucess, the program returns \texttt{0}.

\subsubsection{ToDo}
\begin{tabular}{l l l}
Date & What & Status\\
\hline \\
11,Jun,15  & add switch for excluding certain stations & Waiting ...\\
\end{tabular}

\subsubsection{Bugs}
Send reports to:\\
Xanthos Papanikolaou \href{mailto:xanthos@mail.ntua.gr}{mailto:xanthos@mail.ntua.gr}\\
Demitris Anastasiou  \href{mailto:danast@mail.ntua.gr}{mailto:danast@mail.ntua.gr}\\
Vangelis Zacharis  \href{mailto:vanzach@survey.ntua.gr}{mailto:vanzach@survey.ntua.gr}\\
\bigskip

\begin{tabular}{l l l}
Date & What & Status\\
\hline \\
11,Jun,15  & Add help switch & Waiting ...\\
\end{tabular}