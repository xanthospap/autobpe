% Generated by Sphinx.
\def\sphinxdocclass{report}
\documentclass[letterpaper,10pt,english]{sphinxmanual}
\usepackage[utf8]{inputenc}
\DeclareUnicodeCharacter{00A0}{\nobreakspace}
\usepackage{cmap}
\usepackage[T1]{fontenc}
\usepackage{babel}
\usepackage{times}
\usepackage[Bjarne]{fncychap}
\usepackage{longtable}
\usepackage{sphinx}
\usepackage{multirow}


\title{bernutils Documentation}
\date{July 08, 2015}
\release{0.1}
\author{xp,da}
\newcommand{\sphinxlogo}{}
\renewcommand{\releasename}{Release}
\makeindex

\makeatletter
\def\PYG@reset{\let\PYG@it=\relax \let\PYG@bf=\relax%
    \let\PYG@ul=\relax \let\PYG@tc=\relax%
    \let\PYG@bc=\relax \let\PYG@ff=\relax}
\def\PYG@tok#1{\csname PYG@tok@#1\endcsname}
\def\PYG@toks#1+{\ifx\relax#1\empty\else%
    \PYG@tok{#1}\expandafter\PYG@toks\fi}
\def\PYG@do#1{\PYG@bc{\PYG@tc{\PYG@ul{%
    \PYG@it{\PYG@bf{\PYG@ff{#1}}}}}}}
\def\PYG#1#2{\PYG@reset\PYG@toks#1+\relax+\PYG@do{#2}}

\expandafter\def\csname PYG@tok@gd\endcsname{\def\PYG@tc##1{\textcolor[rgb]{0.63,0.00,0.00}{##1}}}
\expandafter\def\csname PYG@tok@gu\endcsname{\let\PYG@bf=\textbf\def\PYG@tc##1{\textcolor[rgb]{0.50,0.00,0.50}{##1}}}
\expandafter\def\csname PYG@tok@gt\endcsname{\def\PYG@tc##1{\textcolor[rgb]{0.00,0.27,0.87}{##1}}}
\expandafter\def\csname PYG@tok@gs\endcsname{\let\PYG@bf=\textbf}
\expandafter\def\csname PYG@tok@gr\endcsname{\def\PYG@tc##1{\textcolor[rgb]{1.00,0.00,0.00}{##1}}}
\expandafter\def\csname PYG@tok@cm\endcsname{\let\PYG@it=\textit\def\PYG@tc##1{\textcolor[rgb]{0.25,0.50,0.56}{##1}}}
\expandafter\def\csname PYG@tok@vg\endcsname{\def\PYG@tc##1{\textcolor[rgb]{0.73,0.38,0.84}{##1}}}
\expandafter\def\csname PYG@tok@m\endcsname{\def\PYG@tc##1{\textcolor[rgb]{0.13,0.50,0.31}{##1}}}
\expandafter\def\csname PYG@tok@mh\endcsname{\def\PYG@tc##1{\textcolor[rgb]{0.13,0.50,0.31}{##1}}}
\expandafter\def\csname PYG@tok@cs\endcsname{\def\PYG@tc##1{\textcolor[rgb]{0.25,0.50,0.56}{##1}}\def\PYG@bc##1{\setlength{\fboxsep}{0pt}\colorbox[rgb]{1.00,0.94,0.94}{\strut ##1}}}
\expandafter\def\csname PYG@tok@ge\endcsname{\let\PYG@it=\textit}
\expandafter\def\csname PYG@tok@vc\endcsname{\def\PYG@tc##1{\textcolor[rgb]{0.73,0.38,0.84}{##1}}}
\expandafter\def\csname PYG@tok@il\endcsname{\def\PYG@tc##1{\textcolor[rgb]{0.13,0.50,0.31}{##1}}}
\expandafter\def\csname PYG@tok@go\endcsname{\def\PYG@tc##1{\textcolor[rgb]{0.20,0.20,0.20}{##1}}}
\expandafter\def\csname PYG@tok@cp\endcsname{\def\PYG@tc##1{\textcolor[rgb]{0.00,0.44,0.13}{##1}}}
\expandafter\def\csname PYG@tok@gi\endcsname{\def\PYG@tc##1{\textcolor[rgb]{0.00,0.63,0.00}{##1}}}
\expandafter\def\csname PYG@tok@gh\endcsname{\let\PYG@bf=\textbf\def\PYG@tc##1{\textcolor[rgb]{0.00,0.00,0.50}{##1}}}
\expandafter\def\csname PYG@tok@ni\endcsname{\let\PYG@bf=\textbf\def\PYG@tc##1{\textcolor[rgb]{0.84,0.33,0.22}{##1}}}
\expandafter\def\csname PYG@tok@nl\endcsname{\let\PYG@bf=\textbf\def\PYG@tc##1{\textcolor[rgb]{0.00,0.13,0.44}{##1}}}
\expandafter\def\csname PYG@tok@nn\endcsname{\let\PYG@bf=\textbf\def\PYG@tc##1{\textcolor[rgb]{0.05,0.52,0.71}{##1}}}
\expandafter\def\csname PYG@tok@no\endcsname{\def\PYG@tc##1{\textcolor[rgb]{0.38,0.68,0.84}{##1}}}
\expandafter\def\csname PYG@tok@na\endcsname{\def\PYG@tc##1{\textcolor[rgb]{0.25,0.44,0.63}{##1}}}
\expandafter\def\csname PYG@tok@nb\endcsname{\def\PYG@tc##1{\textcolor[rgb]{0.00,0.44,0.13}{##1}}}
\expandafter\def\csname PYG@tok@nc\endcsname{\let\PYG@bf=\textbf\def\PYG@tc##1{\textcolor[rgb]{0.05,0.52,0.71}{##1}}}
\expandafter\def\csname PYG@tok@nd\endcsname{\let\PYG@bf=\textbf\def\PYG@tc##1{\textcolor[rgb]{0.33,0.33,0.33}{##1}}}
\expandafter\def\csname PYG@tok@ne\endcsname{\def\PYG@tc##1{\textcolor[rgb]{0.00,0.44,0.13}{##1}}}
\expandafter\def\csname PYG@tok@nf\endcsname{\def\PYG@tc##1{\textcolor[rgb]{0.02,0.16,0.49}{##1}}}
\expandafter\def\csname PYG@tok@si\endcsname{\let\PYG@it=\textit\def\PYG@tc##1{\textcolor[rgb]{0.44,0.63,0.82}{##1}}}
\expandafter\def\csname PYG@tok@s2\endcsname{\def\PYG@tc##1{\textcolor[rgb]{0.25,0.44,0.63}{##1}}}
\expandafter\def\csname PYG@tok@vi\endcsname{\def\PYG@tc##1{\textcolor[rgb]{0.73,0.38,0.84}{##1}}}
\expandafter\def\csname PYG@tok@nt\endcsname{\let\PYG@bf=\textbf\def\PYG@tc##1{\textcolor[rgb]{0.02,0.16,0.45}{##1}}}
\expandafter\def\csname PYG@tok@nv\endcsname{\def\PYG@tc##1{\textcolor[rgb]{0.73,0.38,0.84}{##1}}}
\expandafter\def\csname PYG@tok@s1\endcsname{\def\PYG@tc##1{\textcolor[rgb]{0.25,0.44,0.63}{##1}}}
\expandafter\def\csname PYG@tok@gp\endcsname{\let\PYG@bf=\textbf\def\PYG@tc##1{\textcolor[rgb]{0.78,0.36,0.04}{##1}}}
\expandafter\def\csname PYG@tok@sh\endcsname{\def\PYG@tc##1{\textcolor[rgb]{0.25,0.44,0.63}{##1}}}
\expandafter\def\csname PYG@tok@ow\endcsname{\let\PYG@bf=\textbf\def\PYG@tc##1{\textcolor[rgb]{0.00,0.44,0.13}{##1}}}
\expandafter\def\csname PYG@tok@sx\endcsname{\def\PYG@tc##1{\textcolor[rgb]{0.78,0.36,0.04}{##1}}}
\expandafter\def\csname PYG@tok@bp\endcsname{\def\PYG@tc##1{\textcolor[rgb]{0.00,0.44,0.13}{##1}}}
\expandafter\def\csname PYG@tok@c1\endcsname{\let\PYG@it=\textit\def\PYG@tc##1{\textcolor[rgb]{0.25,0.50,0.56}{##1}}}
\expandafter\def\csname PYG@tok@kc\endcsname{\let\PYG@bf=\textbf\def\PYG@tc##1{\textcolor[rgb]{0.00,0.44,0.13}{##1}}}
\expandafter\def\csname PYG@tok@c\endcsname{\let\PYG@it=\textit\def\PYG@tc##1{\textcolor[rgb]{0.25,0.50,0.56}{##1}}}
\expandafter\def\csname PYG@tok@mf\endcsname{\def\PYG@tc##1{\textcolor[rgb]{0.13,0.50,0.31}{##1}}}
\expandafter\def\csname PYG@tok@err\endcsname{\def\PYG@bc##1{\setlength{\fboxsep}{0pt}\fcolorbox[rgb]{1.00,0.00,0.00}{1,1,1}{\strut ##1}}}
\expandafter\def\csname PYG@tok@kd\endcsname{\let\PYG@bf=\textbf\def\PYG@tc##1{\textcolor[rgb]{0.00,0.44,0.13}{##1}}}
\expandafter\def\csname PYG@tok@ss\endcsname{\def\PYG@tc##1{\textcolor[rgb]{0.32,0.47,0.09}{##1}}}
\expandafter\def\csname PYG@tok@sr\endcsname{\def\PYG@tc##1{\textcolor[rgb]{0.14,0.33,0.53}{##1}}}
\expandafter\def\csname PYG@tok@mo\endcsname{\def\PYG@tc##1{\textcolor[rgb]{0.13,0.50,0.31}{##1}}}
\expandafter\def\csname PYG@tok@mi\endcsname{\def\PYG@tc##1{\textcolor[rgb]{0.13,0.50,0.31}{##1}}}
\expandafter\def\csname PYG@tok@kn\endcsname{\let\PYG@bf=\textbf\def\PYG@tc##1{\textcolor[rgb]{0.00,0.44,0.13}{##1}}}
\expandafter\def\csname PYG@tok@o\endcsname{\def\PYG@tc##1{\textcolor[rgb]{0.40,0.40,0.40}{##1}}}
\expandafter\def\csname PYG@tok@kr\endcsname{\let\PYG@bf=\textbf\def\PYG@tc##1{\textcolor[rgb]{0.00,0.44,0.13}{##1}}}
\expandafter\def\csname PYG@tok@s\endcsname{\def\PYG@tc##1{\textcolor[rgb]{0.25,0.44,0.63}{##1}}}
\expandafter\def\csname PYG@tok@kp\endcsname{\def\PYG@tc##1{\textcolor[rgb]{0.00,0.44,0.13}{##1}}}
\expandafter\def\csname PYG@tok@w\endcsname{\def\PYG@tc##1{\textcolor[rgb]{0.73,0.73,0.73}{##1}}}
\expandafter\def\csname PYG@tok@kt\endcsname{\def\PYG@tc##1{\textcolor[rgb]{0.56,0.13,0.00}{##1}}}
\expandafter\def\csname PYG@tok@sc\endcsname{\def\PYG@tc##1{\textcolor[rgb]{0.25,0.44,0.63}{##1}}}
\expandafter\def\csname PYG@tok@sb\endcsname{\def\PYG@tc##1{\textcolor[rgb]{0.25,0.44,0.63}{##1}}}
\expandafter\def\csname PYG@tok@k\endcsname{\let\PYG@bf=\textbf\def\PYG@tc##1{\textcolor[rgb]{0.00,0.44,0.13}{##1}}}
\expandafter\def\csname PYG@tok@se\endcsname{\let\PYG@bf=\textbf\def\PYG@tc##1{\textcolor[rgb]{0.25,0.44,0.63}{##1}}}
\expandafter\def\csname PYG@tok@sd\endcsname{\let\PYG@it=\textit\def\PYG@tc##1{\textcolor[rgb]{0.25,0.44,0.63}{##1}}}

\def\PYGZbs{\char`\\}
\def\PYGZus{\char`\_}
\def\PYGZob{\char`\{}
\def\PYGZcb{\char`\}}
\def\PYGZca{\char`\^}
\def\PYGZam{\char`\&}
\def\PYGZlt{\char`\<}
\def\PYGZgt{\char`\>}
\def\PYGZsh{\char`\#}
\def\PYGZpc{\char`\%}
\def\PYGZdl{\char`\$}
\def\PYGZhy{\char`\-}
\def\PYGZsq{\char`\'}
\def\PYGZdq{\char`\"}
\def\PYGZti{\char`\~}
% for compatibility with earlier versions
\def\PYGZat{@}
\def\PYGZlb{[}
\def\PYGZrb{]}
\makeatother

\renewcommand\PYGZsq{\textquotesingle}

\begin{document}

\maketitle
\tableofcontents
\phantomsection\label{index::doc}


Contents:


\chapter{bernutils Introduction}
\label{intro:bernutils-introduction}\label{intro:welcome-to-bernutils-s-documentation}\label{intro::doc}
\textbf{National Technical University of Athens}
\emph{Dionysos Satellite Observatory}
\emph{Higher Geodesy Laboratory}

Documentation for the bernutils Python package

The \emph{bernutils} package is a collection of Python modules to
assist the automatic processing of GNSS data at the
National Technical University of Athens, carried out via the
Bernese v5.2 GNSS Software.


\chapter{Module : bcrd}
\label{bcrd::doc}\label{bcrd:module-bcrd}
This module contains the class \textbf{crdfile} which represents a Bernese v5.2
station coordinate file.

A class named \textbf{crdpoints} is also included, to make easier the handling of
information recorded in the coordinate files.

Station Coordinate files (extension .CRD), contain coordinate components,
names and flags for a list of stations. Corrdinates are always in a
Cartesian reference frame, as {[}x,y,z{]} components, in meters.

Station Information file format format is very strict, and most of the
functions/modules depend that this format is kept. For an example of .CRD
files, see \href{ftp://ftp.unibe.ch/aiub/BSWUSER52/STA/}{ftp://ftp.unibe.ch/aiub/BSWUSER52/STA/} and the collection of .CRD
files placed there.


\section{Documentation}
\label{bcrd:documentation}\index{crdpoint (class in bcrd)}

\begin{fulllineitems}
\phantomsection\label{bcrd:bcrd.crdpoint}\pysiglinewithargsret{\strong{class }\code{bcrd.}\bfcode{crdpoint}}{\emph{name='`}, \emph{number='`}, \emph{xcmp='.0'}, \emph{ycmp='.0'}, \emph{zcmp='.0'}, \emph{flag='`}}{}
Class to represent a (GNSS/geodetic) Point as recorded in a Bernese
format .CRD file.
\index{asString() (bcrd.crdpoint method)}

\begin{fulllineitems}
\phantomsection\label{bcrd:bcrd.crdpoint.asString}\pysiglinewithargsret{\bfcode{asString}}{\emph{aa=1}}{}
Compile a Bernese v5.2 .CRD file record line, using the instance's
attributes. \code{aa} is the number of station (can be any positive 
integer), written at the begining of the returned line.
\begin{quote}\begin{description}
\item[{Returns}] \leavevmode
A .CRD record line.

\end{description}\end{quote}

\end{fulllineitems}

\index{flag\_ (bcrd.crdpoint attribute)}

\begin{fulllineitems}
\phantomsection\label{bcrd:bcrd.crdpoint.flag_}\pysigline{\bfcode{flag\_}\strong{ = `'}}
flag

\end{fulllineitems}

\index{name() (bcrd.crdpoint method)}

\begin{fulllineitems}
\phantomsection\label{bcrd:bcrd.crdpoint.name}\pysiglinewithargsret{\bfcode{name}}{\emph{use\_marker\_number=True}}{}
Return the full name of a station, i.e. 
\code{marker\_name} + \code{marker\_number}.
If \code{use\_marker\_number} is set to \code{False} then the instance's
\code{marker number} will not be included in the name returned.

\end{fulllineitems}

\index{name\_ (bcrd.crdpoint attribute)}

\begin{fulllineitems}
\phantomsection\label{bcrd:bcrd.crdpoint.name_}\pysigline{\bfcode{name\_}\strong{ = `'}}
Station name (4-digit); e.g. `ANKR'

\end{fulllineitems}

\index{number\_ (bcrd.crdpoint attribute)}

\begin{fulllineitems}
\phantomsection\label{bcrd:bcrd.crdpoint.number_}\pysigline{\bfcode{number\_}\strong{ = `'}}
Station number; e.g. `20805M002'

\end{fulllineitems}

\index{setFromCrdLine() (bcrd.crdpoint method)}

\begin{fulllineitems}
\phantomsection\label{bcrd:bcrd.crdpoint.setFromCrdLine}\pysiglinewithargsret{\bfcode{setFromCrdLine}}{\emph{line}}{}
Set (re-initialize) a crdpoint from a .CRD data line. Bernese v5.2
.CRD files have a strict format and this function expects such a
format to be followed.
\begin{quote}\begin{description}
\item[{Parameters}] \leavevmode
\textbf{line} -- A line containing coordinate information, as extracted
(read) from a .CRD file

\item[{Returns}] \leavevmode
Nothing

\end{description}\end{quote}

\end{fulllineitems}

\index{xcmp\_ (bcrd.crdpoint attribute)}

\begin{fulllineitems}
\phantomsection\label{bcrd:bcrd.crdpoint.xcmp_}\pysigline{\bfcode{xcmp\_}\strong{ = 0.0}}
x-component

\end{fulllineitems}

\index{ycmp\_ (bcrd.crdpoint attribute)}

\begin{fulllineitems}
\phantomsection\label{bcrd:bcrd.crdpoint.ycmp_}\pysigline{\bfcode{ycmp\_}\strong{ = 0.0}}
y-component

\end{fulllineitems}

\index{zcmp\_ (bcrd.crdpoint attribute)}

\begin{fulllineitems}
\phantomsection\label{bcrd:bcrd.crdpoint.zcmp_}\pysigline{\bfcode{zcmp\_}\strong{ = 0.0}}
z-component

\end{fulllineitems}


\end{fulllineitems}

\index{crdfile (class in bcrd)}

\begin{fulllineitems}
\phantomsection\label{bcrd:bcrd.crdfile}\pysiglinewithargsret{\strong{class }\code{bcrd.}\bfcode{crdfile}}{\emph{filename}}{}
A class to hold a Bernese v5.2 format .CRD file.
\index{filename\_ (bcrd.crdfile attribute)}

\begin{fulllineitems}
\phantomsection\label{bcrd:bcrd.crdfile.filename_}\pysigline{\bfcode{filename\_}\strong{ = `'}}
the name of the file

\end{fulllineitems}

\index{getFileHeader() (bcrd.crdfile method)}

\begin{fulllineitems}
\phantomsection\label{bcrd:bcrd.crdfile.getFileHeader}\pysiglinewithargsret{\bfcode{getFileHeader}}{}{}
Return the header of a .CRD file as a list of lines,
with no trailing newline chars.
\begin{quote}\begin{description}
\item[{Returns}] \leavevmode
A list of lines included in the file header.

\end{description}\end{quote}

\end{fulllineitems}

\index{getListOfPoints() (bcrd.crdfile method)}

\begin{fulllineitems}
\phantomsection\label{bcrd:bcrd.crdfile.getListOfPoints}\pysiglinewithargsret{\bfcode{getListOfPoints}}{\emph{stalst=None}, \emph{disregard\_number=False}}{}
Read points off from a .CRD file; return all points as list.
If the optional argument \code{stalst} is given, (which is supposed
to hold a list of station names), then only stations matched
in the \code{stalst} list will be returned. By matched, i mean that
the tuple \code{(marker\_name, marker\_number)} is the same for both
stations.
If \code{disregard\_number} is set to \code{True} and \code{stalst} is other 
than \code{None}, then the comparisson of station names, will only 
be performed using the 4char station id (i.e. \code{self.name\_}) and 
\emph{NOT} the marker number.
\begin{quote}\begin{description}
\item[{Returns}] \leavevmode
A list of points (i.e. \code{crdpoint} s)

\end{description}\end{quote}

\end{fulllineitems}


\end{fulllineitems}



\section{Examples}
\label{bcrd:examples}

\chapter{Module : bgps}
\label{bgps:module-bgps}\label{bgps::doc}
This module contains the class \textbf{gpsoutfile} which represents a Bernese v5.2
GPSEST output file.


\section{Documentation}
\label{bgps:documentation}\index{gpsoutfile (class in bgps)}

\begin{fulllineitems}
\phantomsection\label{bgps:bgps.gpsoutfile}\pysiglinewithargsret{\strong{class }\code{bgps.}\bfcode{gpsoutfile}}{\emph{filename}}{}
A class to hold a Bernese v5.2 GPSEST output file
\index{findFirstLine() (bgps.gpsoutfile method)}

\begin{fulllineitems}
\phantomsection\label{bgps:bgps.gpsoutfile.findFirstLine}\pysiglinewithargsret{\bfcode{findFirstLine}}{\emph{stream}, \emph{line}, \emph{eof\_line='\textgreater{}\textgreater{}\textgreater{}'}, \emph{max\_lines=1000}}{}
Given a GPSEST output file, try to match the line passed as
\code{line}, until \code{eof\_line} is not matched and no more than
\code{max\_lines} are read.
If the line is found, the stream input position, will be returned
at the end of the matched line.
\begin{quote}\begin{description}
\item[{Parameters}] \leavevmode\begin{itemize}
\item {} 
\textbf{stream} -- The (calling) instance's input stream.

\item {} 
\textbf{line} -- The prototype line to match.

\item {} 
\textbf{eof\_line} -- EOF record.

\item {} 
\textbf{max\_lines} -- Max lines to be read befor quiting.

\end{itemize}

\item[{Returns}] \leavevmode
On success, the matched line; input buffer is set
at the end of the matched line.

\item[{Warning}] \leavevmode
Note that the search will start from the current get 
position of the stream and \emph{NOT} from the begining of 
the file.

\end{description}\end{quote}

\end{fulllineitems}

\index{getBaselineList() (bgps.gpsoutfile method)}

\begin{fulllineitems}
\phantomsection\label{bgps:bgps.gpsoutfile.getBaselineList}\pysiglinewithargsret{\bfcode{getBaselineList}}{}{}
This function will try to read all baselines processed in the run, 
and report their information. The baselines are read from the 
section:
\emph{2. OBSERVATION FILES, MAIN CHARACTERISTICS}
Two tables are read, namely:
{[}FILE  OBSERVATION FILE HEADER          OBSERVATION FILE                  SESS     RECEIVER 1            RECEIVER 2{]}
and
{[}FILE TYP FREQ.  STATION 1        STATION 2        SESS  FIRST OBSERV.TIME  \#EPO  DT \#EF \#CLK ARC \#SAT  W 12    \#AMB  L1  L2  L5  RM{]}
and the concatenated list is returned, i.e.:
{[}aa, baseline\_name, type, frequency, station1, station2, first\_observation, \# of epochs{]}

\end{fulllineitems}

\index{getCrdSolInfo() (bgps.gpsoutfile method)}

\begin{fulllineitems}
\phantomsection\label{bgps:bgps.gpsoutfile.getCrdSolInfo}\pysiglinewithargsret{\bfcode{getCrdSolInfo}}{}{}
Given a GPSEST output file, this function will try to read information regarding the
(solution) coordinate results. The information is collected from the table:
`NUM  STATION NAME     PARAMETER    A PRIORI VALUE       NEW VALUE     NEW- A PRIORI  RMS ERROR   3-D ELLIPSOID       2-D ELLIPSE'
for every stations already mentioned in the instances station list. So make sure,
the instances station list is already filled. The return list, contains
a list for every station, in the following format:
{[}name,3x(a-priori,estimated,new-old,rms),3x(new-old,rms){]}
for   X, Y, Z                            HGT, LAT, LON
TODO A same block of information maybe available in the section `RESULTS PART 2'. Try reading that
before reading the blok from `RESULTS PART 1'.

\end{fulllineitems}

\index{getHeaderInfo() (bgps.gpsoutfile method)}

\begin{fulllineitems}
\phantomsection\label{bgps:bgps.gpsoutfile.getHeaderInfo}\pysiglinewithargsret{\bfcode{getHeaderInfo}}{}{}
Given a GPSEST output file, try to read the header
information and return in in a list as:
campaign\_name, doy, session, year, date\_run, username, \# of stations

\end{fulllineitems}


\end{fulllineitems}



\section{Examples}
\label{bgps:examples}

\chapter{Module : bsta}
\label{bsta:module-bsta}\label{bsta::doc}
This module contains the class \textbf{stafile} which represents a Bernese v5.2
station information file.

Station Information files (extension .STA), contain information for GPS/GNSS
stations regarding naming, equipment, etc, all of which are related to certain
time intervals (epochs). In most cases when using the functions/modules of
a \textbf{stafile} instance to collect information, an epoch must be supplied (in
addition of-course to a station name).

Station Information file format format is very strict, and most of the
functions/modules depend that this format is kept. For an example of .STA
files, see \href{ftp://ftp.unibe.ch/aiub/BSWUSER52/STA/}{ftp://ftp.unibe.ch/aiub/BSWUSER52/STA/} and the collection of .STA
files placed there.


\section{Documentation}
\label{bsta:documentation}\index{stafile (class in bsta)}

\begin{fulllineitems}
\phantomsection\label{bsta:bsta.stafile}\pysiglinewithargsret{\strong{class }\code{bsta.}\bfcode{stafile}}{\emph{filename}}{}
A station information class, to represent Bernese v5.2 format .STA 
files.
\index{findStationType01() (bsta.stafile method)}

\begin{fulllineitems}
\phantomsection\label{bsta:bsta.stafile.findStationType01}\pysiglinewithargsret{\bfcode{findStationType01}}{\emph{station}, \emph{epoch=None}}{}
This function will search for the station information (concerning
a specific station) included in the \emph{`TYPE 01'} block.
Given a station name (e.g. `ANKR' or `ANKR 20805M002'), it will
try to match the information line provided in the \emph{`TYPE 01'} block,
with a flag of \emph{`001'} and \textbf{NOT} \emph{`003'}. The station matching is \textbf{NOT}
performed with the column \emph{`STATION NAME'}, but with \emph{`OLD STATION NAME'}
In some cases, it might be necessary to also suply the epoch for
which the info is needed (some times the stations are renamed and
different names are adopted previous before and after a certain
epoch).
\begin{quote}\begin{description}
\item[{Parameters}] \leavevmode\begin{itemize}
\item {} 
\textbf{station} -- The name of the station to match.

\item {} 
\textbf{epoch} -- A \code{datetime} object, epoch for which the info is
wanted.

\end{itemize}

\item[{Returns}] \leavevmode
The \emph{`TYPE 01'} block record for this station (for
this epoch).

\end{description}\end{quote}
\begin{description}
\item[{\emph{TODO}}] \leavevmode{[}If a renaming line (flag 003) is encountered it is skipped.{]}
What should i do with this line ??

\end{description}

\end{fulllineitems}

\index{findStationType02() (bsta.stafile method)}

\begin{fulllineitems}
\phantomsection\label{bsta:bsta.stafile.findStationType02}\pysiglinewithargsret{\bfcode{findStationType02}}{\emph{station}, \emph{epoch=None}}{}
This function will search for station info recorded in \emph{`TYPE 002'}
and return it.
The station name provided, will be resolved using the \emph{`TYPE 001'}
block using the function \code{findStationType01}. Hence, the name used to
match info in the \emph{`TYPE 002'} block will be the column \emph{`STATION NAME'}
E.g., given station name `ANKR' and using the CODE.STA file, we
will get the name `ANKR 20805M002' and we will search the block
`TYPE 002' for this name. Note that in case of renaming, a specific
date may be needed (see \code{findStationType01}).
If no date is provided, all entried for the station will be matched
and returned; else, only the one withing the specified interval
will be returned.
\begin{quote}\begin{description}
\item[{Parameters}] \leavevmode\begin{itemize}
\item {} 
\textbf{station} -- The name of the station to match.

\item {} 
\textbf{epoch} -- A \code{datetime} object, epoch for which the info is
wanted.

\end{itemize}

\item[{Returns}] \leavevmode
A list of \emph{`TYPE 02'} block records for this station
(for this epoch).

\end{description}\end{quote}

\end{fulllineitems}

\index{findTypeStart() (bsta.stafile method)}

\begin{fulllineitems}
\phantomsection\label{bsta:bsta.stafile.findTypeStart}\pysiglinewithargsret{\bfcode{findTypeStart}}{\emph{stream}, \emph{type}, \emph{max\_lines=1000}}{}
Given a (sta) input file stream, go to the line where a specific
type starts. E.g. if the type specified is \emph{`1'}, then this function
will search for the line: \emph{`TYPE 001: RENAMING OF STATIONS'}.
The line to search for, is compiled using the \code{type} and the 
\code{\_\_type\_names} dictionary. If the line is not found after 
\code{max\_lines} are read, then an exception is thrown.
In case of sucess, the (header) line is returned, and the stream
buffer is placed at the end of the header line.
Note that the \code{type} parameter must be a positive integer in the
range \code{{[}1, len(\_\_\_type\_names){]}}.
\begin{quote}\begin{description}
\item[{Parameters}] \leavevmode\begin{itemize}
\item {} 
\textbf{stream} -- The input stream for this istance.

\item {} 
\textbf{type} -- The number of type to search for.

\item {} 
\textbf{max\_lines} -- Max lines to read before quiting.

\end{itemize}

\item[{Returns}] \leavevmode
On sucess, the matched line; the input stream is 
left at the end of the matched line.

\end{description}\end{quote}

\end{fulllineitems}

\index{getStationAntenna() (bsta.stafile method)}

\begin{fulllineitems}
\phantomsection\label{bsta:bsta.stafile.getStationAntenna}\pysiglinewithargsret{\bfcode{getStationAntenna}}{\emph{station}, \emph{epoch=None}}{}
Find and return the antenna type, as recorded in the \emph{`TYPE 002'}
block

\end{fulllineitems}

\index{getStationName() (bsta.stafile method)}

\begin{fulllineitems}
\phantomsection\label{bsta:bsta.stafile.getStationName}\pysiglinewithargsret{\bfcode{getStationName}}{\emph{station}, \emph{epoch=None}}{}
Find and return the station name, as recorded in the \emph{`TYPE 001'}
block

\end{fulllineitems}

\index{getStationReceiver() (bsta.stafile method)}

\begin{fulllineitems}
\phantomsection\label{bsta:bsta.stafile.getStationReceiver}\pysiglinewithargsret{\bfcode{getStationReceiver}}{\emph{station}, \emph{epoch=None}}{}
Find and return the receiver type, as recorded in the \emph{`TYPE 002'}
block

\end{fulllineitems}


\end{fulllineitems}



\section{Examples}
\label{bsta:examples}
Example usage of the class \textbf{stafile}, using CODE's .STA file
(available at \textless{}\href{ftp://ftp.unibe.ch/aiub/BSWUSER52/STA/CODE.STA}{ftp://ftp.unibe.ch/aiub/BSWUSER52/STA/CODE.STA}\textgreater{})

\begin{Verbatim}[commandchars=\\\{\}]
\PYG{c}{\PYGZsh{}\PYGZsh{} Test Program}
\PYG{n}{x} \PYG{o}{=} \PYG{n}{StaFile}\PYG{p}{(}\PYG{l+s}{\PYGZsq{}}\PYG{l+s}{CODE.STA}\PYG{l+s}{\PYGZsq{}}\PYG{p}{)}
\PYG{c}{\PYGZsh{}\PYGZsh{} a renaming takes place, so this will fail if no epoch is given}
\PYG{n}{ln1} \PYG{o}{=} \PYG{n}{x}\PYG{o}{.}\PYG{n}{findStationType01}\PYG{p}{(}\PYG{l+s}{\PYGZsq{}}\PYG{l+s}{S071}\PYG{l+s}{\PYGZsq{}}\PYG{p}{,}\PYG{n}{datetime}\PYG{o}{.}\PYG{n}{datetime}\PYG{p}{(}\PYG{l+m+mi}{2008}\PYG{p}{,}\PYG{l+m+mo}{01}\PYG{p}{,}\PYG{l+m+mo}{01}\PYG{p}{,}\PYG{l+m+mo}{01}\PYG{p}{,}\PYG{l+m+mo}{00}\PYG{p}{,}\PYG{l+m+mo}{00}\PYG{p}{)}\PYG{p}{)}
\PYG{k}{print} \PYG{n}{ln1}
\PYG{c}{\PYGZsh{}\PYGZsh{} a renaming takes place, so this will fail if no epoch is given}
\PYG{n}{ln1} \PYG{o}{=} \PYG{n}{x}\PYG{o}{.}\PYG{n}{findStationType01}\PYG{p}{(}\PYG{l+s}{\PYGZsq{}}\PYG{l+s}{S071}\PYG{l+s}{\PYGZsq{}}\PYG{p}{,}\PYG{n}{datetime}\PYG{o}{.}\PYG{n}{datetime}\PYG{p}{(}\PYG{l+m+mi}{2005}\PYG{p}{,}\PYG{l+m+mo}{01}\PYG{p}{,}\PYG{l+m+mo}{01}\PYG{p}{,}\PYG{l+m+mo}{01}\PYG{p}{,}\PYG{l+m+mo}{00}\PYG{p}{,}\PYG{l+m+mo}{00}\PYG{p}{)}\PYG{p}{)}
\PYG{k}{print} \PYG{n}{ln1}
\PYG{c}{\PYGZsh{}\PYGZsh{} following two should be the same}
\PYG{n}{ln1} \PYG{o}{=} \PYG{n}{x}\PYG{o}{.}\PYG{n}{findStationType01}\PYG{p}{(}\PYG{l+s}{\PYGZsq{}}\PYG{l+s}{OSN1 23904S001}\PYG{l+s}{\PYGZsq{}}\PYG{p}{,}\PYG{n}{datetime}\PYG{o}{.}\PYG{n}{datetime}\PYG{p}{(}\PYG{l+m+mi}{2005}\PYG{p}{,}\PYG{l+m+mo}{01}\PYG{p}{,}\PYG{l+m+mo}{01}\PYG{p}{,}\PYG{l+m+mo}{01}\PYG{p}{,}\PYG{l+m+mo}{00}\PYG{p}{,}\PYG{l+m+mo}{00}\PYG{p}{)}\PYG{p}{)}
\PYG{k}{print} \PYG{n}{ln1}
\PYG{n}{ln1} \PYG{o}{=} \PYG{n}{x}\PYG{o}{.}\PYG{n}{findStationType01}\PYG{p}{(}\PYG{l+s}{\PYGZsq{}}\PYG{l+s}{OSN1 23904S001}\PYG{l+s}{\PYGZsq{}}\PYG{p}{)}
\PYG{k}{print} \PYG{n}{ln1}
\PYG{c}{\PYGZsh{}\PYGZsh{} a renaming takes place, so this will fail if no epoch is given}
\PYG{n}{ln2} \PYG{o}{=} \PYG{n}{x}\PYG{o}{.}\PYG{n}{findStationType02}\PYG{p}{(}\PYG{l+s}{\PYGZsq{}}\PYG{l+s}{S071}\PYG{l+s}{\PYGZsq{}}\PYG{p}{,}\PYG{n}{datetime}\PYG{o}{.}\PYG{n}{datetime}\PYG{p}{(}\PYG{l+m+mi}{2005}\PYG{p}{,}\PYG{l+m+mo}{01}\PYG{p}{,}\PYG{l+m+mo}{01}\PYG{p}{,}\PYG{l+m+mo}{01}\PYG{p}{,}\PYG{l+m+mo}{00}\PYG{p}{,}\PYG{l+m+mo}{00}\PYG{p}{)}\PYG{p}{)}
\PYG{k}{print} \PYG{n}{ln2}
\PYG{c}{\PYGZsh{}\PYGZsh{} return all entries, for all epochs}
\PYG{n}{ln2} \PYG{o}{=} \PYG{n}{x}\PYG{o}{.}\PYG{n}{findStationType02}\PYG{p}{(}\PYG{l+s}{\PYGZsq{}}\PYG{l+s}{ANKR}\PYG{l+s}{\PYGZsq{}}\PYG{p}{)}
\PYG{k}{print} \PYG{n}{ln2}
\PYG{c}{\PYGZsh{}\PYGZsh{} return entry for a specific interval}
\PYG{n}{ln2} \PYG{o}{=} \PYG{n}{x}\PYG{o}{.}\PYG{n}{findStationType02}\PYG{p}{(}\PYG{l+s}{\PYGZsq{}}\PYG{l+s}{ANKR}\PYG{l+s}{\PYGZsq{}}\PYG{p}{,}\PYG{n}{datetime}\PYG{o}{.}\PYG{n}{datetime}\PYG{p}{(}\PYG{l+m+mi}{2015}\PYG{p}{,}\PYG{l+m+mo}{07}\PYG{p}{,}\PYG{l+m+mo}{01}\PYG{p}{,}\PYG{l+m+mo}{01}\PYG{p}{,}\PYG{l+m+mo}{00}\PYG{p}{,}\PYG{l+m+mo}{00}\PYG{p}{)}\PYG{p}{)}
\PYG{k}{print} \PYG{n}{ln2}
\PYG{k}{print} \PYG{n}{x}\PYG{o}{.}\PYG{n}{getStationName}\PYG{p}{(}\PYG{l+s}{\PYGZsq{}}\PYG{l+s}{S071}\PYG{l+s}{\PYGZsq{}}\PYG{p}{,}\PYG{n}{datetime}\PYG{o}{.}\PYG{n}{datetime}\PYG{p}{(}\PYG{l+m+mi}{2008}\PYG{p}{,}\PYG{l+m+mo}{01}\PYG{p}{,}\PYG{l+m+mo}{01}\PYG{p}{,}\PYG{l+m+mo}{01}\PYG{p}{,}\PYG{l+m+mo}{00}\PYG{p}{,}\PYG{l+m+mo}{00}\PYG{p}{)}\PYG{p}{)}
\PYG{k}{print} \PYG{n}{x}\PYG{o}{.}\PYG{n}{getStationName}\PYG{p}{(}\PYG{l+s}{\PYGZsq{}}\PYG{l+s}{S071}\PYG{l+s}{\PYGZsq{}}\PYG{p}{,}\PYG{n}{datetime}\PYG{o}{.}\PYG{n}{datetime}\PYG{p}{(}\PYG{l+m+mi}{2005}\PYG{p}{,}\PYG{l+m+mo}{01}\PYG{p}{,}\PYG{l+m+mo}{01}\PYG{p}{,}\PYG{l+m+mo}{01}\PYG{p}{,}\PYG{l+m+mo}{00}\PYG{p}{,}\PYG{l+m+mo}{00}\PYG{p}{)}\PYG{p}{)}
\PYG{k}{print} \PYG{n}{x}\PYG{o}{.}\PYG{n}{getStationName}\PYG{p}{(}\PYG{l+s}{\PYGZsq{}}\PYG{l+s}{ANKR}\PYG{l+s}{\PYGZsq{}}\PYG{p}{)}
\PYG{k}{print} \PYG{n}{x}\PYG{o}{.}\PYG{n}{getStationAntenna}\PYG{p}{(}\PYG{l+s}{\PYGZsq{}}\PYG{l+s}{S071}\PYG{l+s}{\PYGZsq{}}\PYG{p}{,}\PYG{n}{datetime}\PYG{o}{.}\PYG{n}{datetime}\PYG{p}{(}\PYG{l+m+mi}{2008}\PYG{p}{,}\PYG{l+m+mo}{01}\PYG{p}{,}\PYG{l+m+mo}{01}\PYG{p}{,}\PYG{l+m+mo}{01}\PYG{p}{,}\PYG{l+m+mo}{00}\PYG{p}{,}\PYG{l+m+mo}{00}\PYG{p}{)}\PYG{p}{)}
\PYG{k}{print} \PYG{n}{x}\PYG{o}{.}\PYG{n}{getStationAntenna}\PYG{p}{(}\PYG{l+s}{\PYGZsq{}}\PYG{l+s}{S071}\PYG{l+s}{\PYGZsq{}}\PYG{p}{,}\PYG{n}{datetime}\PYG{o}{.}\PYG{n}{datetime}\PYG{p}{(}\PYG{l+m+mi}{2005}\PYG{p}{,}\PYG{l+m+mo}{01}\PYG{p}{,}\PYG{l+m+mo}{01}\PYG{p}{,}\PYG{l+m+mo}{01}\PYG{p}{,}\PYG{l+m+mo}{00}\PYG{p}{,}\PYG{l+m+mo}{00}\PYG{p}{)}\PYG{p}{)}
\PYG{k}{print} \PYG{n}{x}\PYG{o}{.}\PYG{n}{getStationAntenna}\PYG{p}{(}\PYG{l+s}{\PYGZsq{}}\PYG{l+s}{ANKR}\PYG{l+s}{\PYGZsq{}}\PYG{p}{,}\PYG{n}{datetime}\PYG{o}{.}\PYG{n}{datetime}\PYG{p}{(}\PYG{l+m+mi}{2005}\PYG{p}{,}\PYG{l+m+mo}{01}\PYG{p}{,}\PYG{l+m+mo}{01}\PYG{p}{,}\PYG{l+m+mo}{01}\PYG{p}{,}\PYG{l+m+mo}{00}\PYG{p}{,}\PYG{l+m+mo}{00}\PYG{p}{)}\PYG{p}{)}
\PYG{k}{print} \PYG{n}{x}\PYG{o}{.}\PYG{n}{getStationReceiver}\PYG{p}{(}\PYG{l+s}{\PYGZsq{}}\PYG{l+s}{S071}\PYG{l+s}{\PYGZsq{}}\PYG{p}{,}\PYG{n}{datetime}\PYG{o}{.}\PYG{n}{datetime}\PYG{p}{(}\PYG{l+m+mi}{2008}\PYG{p}{,}\PYG{l+m+mo}{01}\PYG{p}{,}\PYG{l+m+mo}{01}\PYG{p}{,}\PYG{l+m+mo}{01}\PYG{p}{,}\PYG{l+m+mo}{00}\PYG{p}{,}\PYG{l+m+mo}{00}\PYG{p}{)}\PYG{p}{)}
\PYG{k}{print} \PYG{n}{x}\PYG{o}{.}\PYG{n}{getStationReceiver}\PYG{p}{(}\PYG{l+s}{\PYGZsq{}}\PYG{l+s}{S071}\PYG{l+s}{\PYGZsq{}}\PYG{p}{,}\PYG{n}{datetime}\PYG{o}{.}\PYG{n}{datetime}\PYG{p}{(}\PYG{l+m+mi}{2005}\PYG{p}{,}\PYG{l+m+mo}{01}\PYG{p}{,}\PYG{l+m+mo}{01}\PYG{p}{,}\PYG{l+m+mo}{01}\PYG{p}{,}\PYG{l+m+mo}{00}\PYG{p}{,}\PYG{l+m+mo}{00}\PYG{p}{)}\PYG{p}{)}
\PYG{k}{print} \PYG{n}{x}\PYG{o}{.}\PYG{n}{getStationReceiver}\PYG{p}{(}\PYG{l+s}{\PYGZsq{}}\PYG{l+s}{ANKR}\PYG{l+s}{\PYGZsq{}}\PYG{p}{,}\PYG{n}{datetime}\PYG{o}{.}\PYG{n}{datetime}\PYG{p}{(}\PYG{l+m+mi}{2005}\PYG{p}{,}\PYG{l+m+mo}{01}\PYG{p}{,}\PYG{l+m+mo}{01}\PYG{p}{,}\PYG{l+m+mo}{01}\PYG{p}{,}\PYG{l+m+mo}{00}\PYG{p}{,}\PYG{l+m+mo}{00}\PYG{p}{)}\PYG{p}{)}
\end{Verbatim}


\chapter{Indices and tables}
\label{index:indices-and-tables}\begin{itemize}
\item {} 
\emph{genindex}

\item {} 
\emph{modindex}

\item {} 
\emph{search}

\end{itemize}



\renewcommand{\indexname}{Index}
\printindex
\end{document}
